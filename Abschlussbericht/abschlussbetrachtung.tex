\section{Abschlussbetrachtung}

\subsection{Veränderung der Projektvorgaben}
Die übersprüngliche Projektbeschreibung sah vor, dass ein Gerätetreiber für eine Kamera entwickelt werden sollte, dabei war allerdings nicht näher spezifiert, wie diese Kamera geartet sein sollte.
Nachdem sich herausstellte, dass die Kamera für das Softwareprojekt nicht verfügbar sein würde, veränderte sich das Projekt dahingehend, dass von jenem Zeitpunkt an ein Gerätetreiber für einen USB-Video-Grabber entwickelt werden sollte.

Die Plattform, für welche der Treiber entwickelt werden sollte, war hierbei das Board \emph{MSBA2}, für welches eine Portierung von RIOT existiert.
Hier tat sich bereits am Anfang die Frage auf, ob das \emph{MSBA2} genug Rechenleistung und Speicher bieten könnte, um entsprechende Puffer zu allozieren; wenig später beantwortete sich jene Frage nach der Einarbeitung in die Programmierung des \emph{MSBA2} bei der Analyse des existierenden Linux-Treibers von selbst.
Es ergab sich, dass das \emph{MSBA2} mit einem Arbeitsspeicher von nur 98\,KiB nicht für die Verarbeitung der Videodaten geeignet war.

Anschließend wurde mit dem \emph{BeagleBone}, einem günstigen Computer auf einer Platine, eine neue Hardwareplattform eingeführt, die wesentlich mehr Rechenleistung und Arbeitsspeicher bot, jedoch änderte sich diese Vorgabe bereits wieder in der nächste Woche.

Die finale Plattform war schließlich der \emph{native port} von RIOT (siehe \autoref{section:native_port}), der praktisch alle Beschränkungen von Rechenleistung und Arbeitsspeicher aufhob und die Möglichkeit, auf eine existierende USB-Funktionalität zurückgreifen zu können, eröffnete.

\subsection{Diskussion des Ergebnisses}
Das Ergebnis dieses Softwareprojektes ist im Grunde genommen ein unfertiger Gerätetreiber, der es zulässt, den \stk{} und den \saa{} zu konfigurieren und welcher tatsächlich einen Videostream bereitstellt, jedoch sind die Videodaten bis zum heutigen Zeitpunkt noch relativ nutzlos, da sie nur bedigt etwas mit der Realität zu tun haben.
\footnote{Helligkeitsunterschiede waren zu erkennen}
Der Hauptgrund dafür ist die Tatsache, dass der verwendete USB-Video-Grabber sowie die Verwendung des \emph{native port} erst gemessen an der Gesamtdauer des Softwareprojektes sehr spät feststand.
Der damit einhergehende effektive Zeitverlust ließ uns zum Ende des Projektes hin nicht mehr die Möglichkeit, die Videodaten einer umfangreichen Analyse zu unterziehen sowie die fehlende Funktionalität, was besonders die Kontrolle des \saa{} betrifft, zu implementieren.

\subsection{Ausblick}
Bis zum produktiven Einsatz des Treibers ist es noch ein relativ weiter Weg, da es noch Baustellen an verschiedenen Stellen gibt:
Es müssen diverse Funktionen des \saa, zum Beispiel Bildverarbeitungen wie Hellkeits- und Kontrastanpassungen oder die Frameratesteuererung, konfigurierbar gemacht werden.
Dazu ist es nötig, das Datenblatt des \saa{} zu studieren und auch den Linuxtreiber weiter zu betrachten.
Ferner ist auch notwendig, die empfangenen Daten zu validieren und an der Interpretation derjenigen zu arbeiten, um überhaupt ein brauchbares Bild zu erhalten.
Zum Abschluss sei noch angemerkt, dass für RIOT noch ein USB-Stack notwendig ist, um gegebenenfalls \libusb{} zu protieren oder gleich die fiktive native USB-Funktionalität von RIOT zu nutzen.

\subsection{Reflexion}
Dieses Softwareprojekt war durch und durch eine interessante Herausforderung und gleichwohl eine hervorragende Möglichkeit, Kenntnisse in vielen Bereichen zu erhalten:
\begin{itemize}
 \item Funktionsweise des USB-Protokolles
 \item Programmierung eines Gerätetreibers
 \item Verwendung von vielen Entwicklungswerkzeugen wie \emph{vim}, \emph{ctags}, \emph{gcc}, \emph{gdb}, \emph{git} und vielen anderen mehr ...
 \item Beflügelung durch Teilerfolge und Demotivation durch Rückschläge
\end{itemize}
Alles in allem ist es eine wertvolle Erfahrung, solch ein Softwareprojekt durchzuführen.
Wir lernten, Arbeit aufzuteilen, zu strukturieren, diese Zweige wieder zusammenzuführen.

Ein bisschen schade ist es aus unserer Sicht am Ende doch, kein Richtiges Bild bekommen zu haben, aber sicherlich wird irgendwann auf Basis dieses Softwareprojektes etwas weiterenwickelt.
