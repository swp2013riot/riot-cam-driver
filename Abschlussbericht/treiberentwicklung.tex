\section{Treiberentwicklung}
\subsection{Anaylse der Linux Treiber}
Für den verwendeten USB-Video-Grabber sind im Linux Kernel bereits zwei 
verschiedene Treiber vorhanden. Da diese Treiber auf ein im Linux Kern 
vorhandenes größeres Video Framework aufsetzten ist ein direktes Portieren der 
entsprechenden Kernel Module nach RIOT nicht ohne weiteres möglich gewesen.
Trotzdem waren die Linux Kernel Module extrem hilfreich bei der Entwicklung
der RIOT Treibers. Unter anderem konnte anhand dieser Module das zusammenspiel
der beiden Chips und die korrekte Ansteuerung selbiger ermittelt werden.
Insbesondere für den Gateway-Chip \stk{} spielten die Linux Treiber eine
besondere Rolle da für diesen Chip kein Datenblatt verfügbar ist. Die Linux 
Treiber waren somit die einzige, uns verfügbare Dokumentation, für diesen Chip. 

\subsubsection{Quellcode-Ebene}
Da es sich bei dem USB-Video-Grabber um einen Clone eines des weit verbreiteten
easycap-USB-Video-Grabber handelt hatten wir zunächst das easycap genannte Kernel
Modul welches in älteren Linux Versionen vorhanden ist betrachtet. Dieser Treiber
ist in modernern Versionen des Linux Kernel (3.x aufwärts) nicht mehr vorhanden.
Ein Grund hierfür ist die sehr schlechte Code-Qualität und die nahezu
nicht vorhandene Dokumentation dieses Moduls. Wegen der schlechten Lesbarkeits des
Codes war der Informationsgewinn durch das betrachten dieses Codes relativ gering.

In aktuelleren Linux Versionen ist sowohl ein Treiber für den \stk{} (Modulname: stk1160) 
wie auch für den \saa{} (Modulname: saa711x) Chip vorhanden. Diese Treiber sind im wesentlichen 
das Ergebnis eines refactoring des alten easycap Moduls. Der Code dieser Module ist wesentlich 
besser lesbar. Die Dokumentation ist aber auch in diesen Modulen eher mangelhaft. Insbesondere 
lassen sich im Code der neuen Module Kommentare finden die darauf schließen lassen, dass der 
Autor einigen aus easycap übernommenen Teile selber nicht verstanden hat.

Trotz dieser Schwierigkeiten konnten wir anhand der neueren Kernel Module das Grundlegende 
Zusammenspiel der beiden beteildigten Chips nachvollziehen.

\subsubsection{Funktionale-Ebene}

\subsubsection{Analyseergebnisse}

\subsection{Aufbau und Funktionsweise des RIOT Treibers}

\subsection{Vorgehen bei der Implementierung}

\subsection{Probleme und Herausforderungen}

\subsection{USB}
Der \textbf{U}niversal \textbf{S}erial \textbf{B}us ist ein serieller Bus, der die Kommunikation von Peripheriegeräten mit dem Computer ermöglicht.
Die Kommunikation läuft immer zwischen zwei Endpunkten ab.
Ein Endpunkt ist eine Art Unteradresse in einem USB-Gerät.
Viele Geräte besitzen Unterfunktionen, denen dann jeweils ein Endpunkt zugeordnet ist.
USB kennt verschiedene Übertragungsmodi:

\paragraph{Synchroner Transfer}
Beim synchronen Transfer (auch Kontrolltransfer genannt) werden durch eine bidirektionale Pipe kurze Datenpakete zwischen zwei Endpunkten hin und her geschickt.
Diese Art des Datentransfers ist speziell für die Konfiguration von Geräten entworfen worden.
